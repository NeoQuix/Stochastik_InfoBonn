\documentclass[a4paper,11pt]{scrartcl}
\usepackage[a4paper, left=2cm, right=2cm, top=2cm, bottom=3cm]{geometry} % kleinere Ränder

% Umlaute in der Datei erlauben, auf Deutsch umstellen
\usepackage[utf8]{inputenc}
\usepackage[ngerman]{babel}

% Mathesymbole und Ähnliches
\usepackage{amsmath}
\usepackage{mathtools}
\usepackage{amssymb}
\usepackage{microtype}
\usepackage{stmaryrd}

% Abbildungen
\usepackage{tikz}
\usetikzlibrary{arrows,calc}

% Bessere Kontrolle über floats
\usepackage{float}

% Aufzählungen anpassen (alternativ: \arabic, \alph)
%\renewcommand{\labelenumi}{(\roman{enumi})}


\begin{document}

% Kopfzeile (nur Nummer der Übung und Namen/MatrNr. müssen verändert werden)
{\raggedright
\begin{tabular}{l}
    Stocha Recap \\
    SS 2021 \\
    \today{}
\end{tabular}}
\hfill
{\Large Kapitel 8: Markov-Ketten}
\hfill
\begin{tabular}{r}
    Spartak Ehrlich \\
    Stocha ist doof
\end{tabular}
\hrule

\section{Grundlagen}
\begin{itemize}
    \item geht um ZVs die eine einfache Form von stochastischer Abhängigkeit haben
    \item und zwar ist die Zufallsvariable $n+1$ nur abhängig von der ZV $n$
    \item bei diesen ZVs interessieren wir uns für das Langzeitkonvergenzverhalten (Markov-Ketten)
\end{itemize}

\section{Stochastische Matrizen und Markov-Ketten}
\subsection{Definition: Stochastische Matrizen}
Sei $ V \neq \emptyset$ und $ \prod = (\prod (x,y))_{x,y \in V}$ eine reellwertige Matrix.

\begin{itemize}
    \item $\prod$ heißt zeilenstochastisch, wenn in jeder Zeile der Matrix $\prod$ eine W-Funktion auf V steht. Das heißt:
    \begin{itemize}
        \item alle Einträge von $\prod$ liegen im Intervall $[0:1]: \prod \in [0:1]^{V \times V}$
        \item für alle $x \in V$ ist $\sum_{y \in V} \prod (x,y) = 1$
    \end{itemize}
\item Wir betrachten den Zufallsprozess in V bei der jedem Schritt die Wahrscheinlichkeit $\prod (x,y)$ vom Zustand $x$ zum Zustand $y$ springt. Symbolisch mit: $x \stackrel{\prod (x,y)}{\rightarrow} y$
\end{itemize}

\subsection{Definition: Markov Kette}
\begin{itemize}
    \item Eine Folge von $X_0,X_1,\dots$ von ZVs auf dem W-Raum $(\Omega,\mathcal{A},P)$ mit Werten in V, d.h. $X_i :(\Omega,\mathcal{A}) \rightarrow (V,2^V)$ heißt \textbf{Markov-Kette mit Zustandsraum V und Übergangmatrix} $\prod$, wenn für alle $n \geq 0$ und für alle $x_0,\dots,x_{n+1} \in V$ folgende Markov Eigenschaft gilt:
        $P(X_{n+1} = x_{n+1}| X_0 = x_0, \dots, X_n = x_n) = P(X_{n+1} = x_{n+1}| X_n =x_n)$ sofern $P(X_0 = x_0, \dots, X_n = x_n) >0$ ist.
    \item Die Verteilung $\alpha := P \circ X^{-1}_0$ von $X_0$ heißt \textbf{Startverteilung der Markov-Kette}
\end{itemize}

\subsection{Satz: Matrixpotenzen}
Die n-te Potenz $\prod^n$ der zeilenstochastischen Matrix $\prod$ enthält an der Position $(x,y)$ die Wahrscheinlichkeit, in genau n Schritten vom Zustand x in den Zustand y zu gelangen:
\begin{itemize}
    \item $P^x(X_n=y)= \prod^n (x,y)$
\end{itemize}

\subsection{Ergodensatz:}
Es sei $\prod = \pi_{i,j} \in [0:1]^{n \times n}$ zeilenstochastisch. Ferner gebe es ein $L \geq 1 $, sodass alle Einträge in $\prod^L$ positiv sind. Dann gibt es eine W-Funktion $p = (p_1,\dots,p_N)$ die sog. Grenzverteilung mit folgenden Eigenschaften:
\begin{itemize}
    \item $\lim_{m \rightarrow \infty} \prod^m =: \prod^\infty$ existiert und in jeder Zeile von $\prod^\infty$ steht die Grenzverteilung.
    \item Die Matrixfolge $(\prod^m)_{m\geq 1}$ konvergiert exponentiell schnell gegen $\prod^\infty$
    \item Die Grenzverteilung $p$ ist die eindeutig bestimme W-Funktion mit $p \prod = p$
\end{itemize}

\end{document}
